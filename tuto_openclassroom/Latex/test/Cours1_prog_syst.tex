\documentclass{report}

\usepackage[latin1]{inputenc}
\usepackage[T1]{fontenc}
\usepackage[francais]{babel}

\begin{document}

\part{Partie}
\chapter{Chapitre}
\section{Section}
\subsection{Une sous-section}
moniteur : on met en parallele l'execution du prog avec un affichage de programme/ lecture 
de carte.
Multi-programmation : pour faire genre à l'utilisateur que  plusieurs programme s'execute en meme temps: le processeur va accorder une petite qtte de temps à chaque programme tres court, du coup on le voit même pas.
Le superviseur charge plusieurs programmes en mémoire :Ilorsque le programme en cours demande une E/S, le superviseur demarre le programme suivantIlorsque le contrôleur d’E/S signale la fin de l’E/S, le premierprogramme reprend son exécution

petit soucis:
pour le processeur lexecution dun user et d'un autre osef.
mais probleme de sécurite dacces des prog dun user a lautre.
on delegue ça au systeme.
si on veut faire des commandes systeme, ca va passer par le systeme qui verifiera si les appels du programme ont le droit ou pas de faire ce quils font.
noyau = ens du code de l'OS.
tiers de confiance : OS.
==> E/S ==>passage en mode privilégié 
        ==>execution dans de code noyau
        ===> verif necessaire ==>ok dans lexe de 
                                ==> ko erreur

systeme exploitation aujourdui: noyau + applications systemes. (genre terminal/ interface graphique, etc)
        
        

\subsection{Une sous-section2}

\paragraph{mon para}
cours fichiers
table des descripteurs propre a chaque programme, chaque ligne correspont a des fichiers ou quoi, et de base ya les 3 descripteurs de base, pour le code sortie on met le 1er chiffre dispo apres les 012 et autre code pris.

table des fichiers ouverts, 3colonnes importantes (puis plein dautre mais osef pour now:
nb entree (code descripteur), mode douverture et loffset.
on clean la ligne quand on ferme le fichier.
lalgo pour faire une ligne prend le 1er entier libre et fait une ligne.
a chaque fois que l'on ouvre un fichier, peut y avoir plusieurs lignes créé, si jamais on enchaine un wreate et trunc par exemple.

lseek renvoie le nb doctect dont la tete s'est déplacée.
\subparagraph{monsubpara}
 non, sagittis non sapien.
Donec interdum pretium venenatis. Pellentesque aliquam convallis convallis.
Fusce tincidunt orci eu velit varius luctus. Etiam iaculis viverra enim ac varius.
Duis pretium elit eu eros auctor vel iaculis nulla commodo. Aliquam interdum fermentum orci sed fringilla.
\end{document}
